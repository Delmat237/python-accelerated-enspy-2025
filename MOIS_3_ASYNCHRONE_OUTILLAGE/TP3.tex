\documentclass[12pt]{article}
\usepackage[utf8]{inputenc}
\usepackage[T1]{fontenc}
\usepackage[a4paper, margin=1in]{geometry}
\usepackage{fancyhdr}
\usepackage{amsmath}
\usepackage{enumitem}
\usepackage{listings}
\usepackage{xcolor}

% Configuration des listings pour le code Python
\lstset{
    language=Python,
    basicstyle=\ttfamily\small,
    keywordstyle=\color{blue},
    stringstyle=\color{red},
    commentstyle=\color{green!50!black},
    numbers=left,
    numberstyle=\tiny,
    stepnumber=1,
    numbersep=5pt,
    showspaces=false,
    showstringspaces=false,
    frame=single,
    breaklines=true,
    breakatwhitespace=true,
    tabsize=4
}

% En-tête et pied de page
\pagestyle{fancy}
\fancyhf{}
\fancyhead[L]{TP 3 - Mois 3 : Python Avancé et Asynchronisme}
\fancyhead[R]{21 janvier 2026}
\fancyfoot[C]{\thepage}

% Titre
\title{\textbf{TP Intégrateur - Décorateurs, Programmation Asynchrone et Outillage Pro}}
\author{}
\date{}

\begin{document}

\maketitle

\section*{ Objectifs du TP (3h)}

Ce TP évalue votre maîtrise des concepts avancés de Python nécessaires pour le développement de systèmes haute performance et la gestion de projets professionnels.

\section{ Décorateurs et Closures : Sécurité et Performance (45 min)}
\textbf{Contexte :} Vous travaillez pour une startup de Fintech à Douala.

\begin{itemize}
    \item Créez une closure \texttt{generer\_id\_transaction(prefixe)} qui retourne une fonction générant des IDs successifs (ex: TX-1, TX-2).
    \item Créez un décorateur \texttt{@verifier\_admin} qui vérifie si une variable globale \texttt{est\_admin} est à \texttt{True} avant d'exécuter une fonction.
    \item Créez un décorateur \texttt{@timeout(secondes)} qui affiche un avertissement si une fonction prend plus de X secondes.
\end{itemize}

\section{ Programmation Asynchrone : Scraper de Prix (1h)}
\textbf{Contexte :} Simulez la récupération de prix de cacao depuis 3 marchés différents en parallèle.

\begin{itemize}
    \item Utilisez \texttt{asyncio} et \texttt{await}.
    \item Créez une fonction \texttt{recuperer\_prix(marche, delai\_simule)}.
    \item Utilisez \texttt{asyncio.gather} pour lancer les 3 requêtes simultanément.
    \item Gérez une erreur simulée (50\% de chances d'échec) avec un bloc \texttt{try/except} asynchrone.
\end{itemize}

\section{ Outillage Avancé : Logging et Configuration (30 min)}
\textbf{Contexte :} Professionnalisez votre script de scraping.

\begin{itemize}
    \item Configurez un \texttt{logger} qui écrit les erreurs dans un fichier \texttt{errors.log} et les infos dans la console.
    \item Utilisez \texttt{python-dotenv} pour charger l'URL de base du marché depuis un fichier \texttt{.env}.
    \item Implémentez un \texttt{sys.excepthook} personnalisé pour logguer toute erreur non gérée.
\end{itemize}

\section{ Multi-threading : Traitement de Données (45 min)}
\textbf{Contexte :} Vous devez transformer 500 images (simulation par \texttt{time.sleep}).

\begin{itemize}
    \item Utilisez \texttt{ThreadPoolExecutor} pour traiter 5 tâches en parallèle.
    \item Comparez le temps d'exécution avec une approche synchrone classique.
\end{itemize}

\section{ Projet Final : Micro-Service de Monitoring (1h)}
\textbf{Consigne :}
Créez un script complet qui :
1. Charge sa configuration depuis un \texttt{.env}.
2. Possède un décorateur de logging pour chaque fonction.
3. Lance une vérification asynchrone de 5 URLs web.
4. Enregistre les résultats dans un fichier de log avec horodatage.

\textbf{Submission Procedure (Git) :} 
Suivre la même procédure que les TPs précédents en utilisant la branche \texttt{tp\_python/tp3-<nom>-<prenom>}.

\end{document}
