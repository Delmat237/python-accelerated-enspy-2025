\documentclass[12pt]{article}
\usepackage[utf8]{inputenc}
\usepackage[T1]{fontenc}
\usepackage[a4paper, margin=1in]{geometry}
\usepackage{fancyhdr}
\usepackage{amsmath}
\usepackage{enumitem}
\usepackage{listings}
\usepackage{xcolor}

% Configuration for Python code listings
\lstset{
    language=Python,
    basicstyle=\ttfamily\small,
    keywordstyle=\color{blue},
    stringstyle=\color{red},
    commentstyle=\color{green!50!black},
    numbers=left,
    numberstyle=\tiny,
    stepnumber=1,
    numbersep=5pt,
    showspaces=false,
    showstringspaces=false,
    frame=single,
    breaklines=true,
    breakatwhitespace=true,
    tabsize=4
}

% Header and Footer
\pagestyle{fancy}
\fancyhf{}
\fancyhead[L]{TP 1 - Month 1: Types, Conditions, Loops, Functions, Structures}
\fancyhead[R]{October 21, 2025}
\fancyfoot[C]{\thepage}

% Title
\title{\textbf{Integrative TP - Concrete Exercises on Types, Conditions, Logic, Loops, Functions and Data Structures}}
\author{}
\date{}

\begin{document}

\maketitle

\section*{TP Objectives (2h - 3h)}

This sheet offers concrete exercises inspired by real everyday problems or engineering fields in Cameroon (e.g., agricultural management, health, local finance, education). The approach is \textbf{practical and code-first}: each exercise simulates a professional scenario to apply the concepts. \\
\textbf{Required Material:} Python 3.11+, IDE (VS Code). Test each exercise by running the code.

\section{Data Types: Real Agricultural Calculations (20 min)}

\subsection{Exercise 1: Seed Cost}
\textbf{Real Context:} In Cameroon, a farmer calculates the cost of seeds for a maize plot. Use types to manage quantities, prices, and descriptions.

\begin{itemize}
    \item Declare \texttt{bags\_quantity} (int), \texttt{price\_per\_bag} (float), \texttt{crop} (str).
    \item Calculate the total (\texttt{bags\_quantity * price\_per\_bag}) and convert it to str with 2 decimal places.
    \item Display a formatted message: "The cost for X bags of Y seeds is Z FCFA."
\end{itemize}

\begin{lstlisting}
bags_quantity = 5  # Number of bags
price_per_bag = 15000.50  # Price per bag in FCFA
crop = "maize"
# Your code here...
total = # Calculation
message = f"The cost for {bags_quantity} bags of {crop} seeds is {total:.2f} FCFA."
print(message)
\end{lstlisting}

\newpage
\subsection{Exercise 2: Unit Conversion}
\begin{itemize}
    \item Convert an area in hectares (float) to m² (int).
    \item Handle an error if the area is negative.
\end{itemize}

\section{2. Conditions and Logic: Social Aid Eligibility (20 min)}

\subsection{Exercise 3: Eligibility Verification}
\textbf{Real Context:} In an aid program for vulnerable families in Cameroon, verify eligibility based on income, number of children, and status (rural/urban).

\begin{itemize}
    \item Ask for \texttt{monthly\_income} (float), \texttt{num\_children} (int), \texttt{is\_rural} (bool).
    \item If income < 100000 FCFA AND num\_children > 2 OR is\_rural, display "Eligible for aid".
    \item Otherwise, "Not eligible".
\end{itemize}

\begin{lstlisting}
monthly_income = float(input("Monthly income (FCFA): "))
num_children = int(input("Number of children: "))
is_rural = input("Rural area? (yes/no) ") == "yes"
# Your code here...
if # Condition:
    print("Eligible for aid")
else:
    print("Not eligible")
\end{lstlisting}

\subsection{Exercise 4: Aid Priority}
\begin{itemize}
    \item Use \texttt{elif} to classify into "High priority" (income < 50000), "Medium" (50000-100000), "Low" (other).
\end{itemize}

\section{3. Loops: Daily Harvest Tracking (30 min)}

\subsection{Exercise 5: Daily Tracking}
\textbf{Real Context:} A farmer tracks the cassava harvest over 7 days, alerting if a day is below average.

\begin{itemize}
    \item Use \texttt{for} with \texttt{range(1, 8)} to simulate days.
    \item Generate a random yield (use \texttt{random.randint(50, 200)} kg/day).
    \item If yield < 100, \texttt{continue}; if < 50, \texttt{break} and alert "Major problem".
\end{itemize}

\begin{lstlisting}
import random
for day in range(1, 8):
    yield_val = random.randint(50, 200)
    if yield_val < 50:
        print(f"Day {day}: Major problem! Stopping.")
        # break
    elif yield_val < 100:
        print(f"Day {day}: Low yield, skipping to next.")
        # continue
    print(f"Day {day}: Yield OK ({yield_val} kg)")
\end{lstlisting}

\subsection{Exercise 6: While Loop for Stock}
\begin{itemize}
    \item Use \texttt{while} to deduct from an initial stock (500 kg) until 0, subtracting daily sales (50-100 kg).
\end{itemize}

\section{4. Functions: Family Budget Tool (30 min)}

\subsection{Exercise 7: Budget Calculation}
\textbf{Real Context:} A Cameroonian family calculates their monthly budget for food, transport, and education.

\begin{itemize}
    \item Define \texttt{calculate_budget(food_expenses, transport_expenses, *other_expenses)} which sums everything and returns the total.
    \item Use \texttt{*args} for other variable expenses.
\end{itemize}

\begin{lstlisting}
def calculate_budget(food_expenses, transport_expenses, *other_expenses):
    total = food_expenses + transport_expenses
    for exp in other_expenses:
        total += exp
    return total

# Test
budget = calculate_budget(100000, 50000, 30000, 20000)  # Food, Transport, Education, Health
print(f"Total budget: {budget} FCFA")
\end{lstlisting}

\subsection{Exercise 8: Function with Scope}
\begin{itemize}
    \item Add a global variable \texttt{subsidy = 20000} and subtract it in the function (without modifying the global).
\end{itemize}

\section{5. Data Structures: Medicine Inventory Management (30 min)}

\subsection{Exercise 9: List for Stock}
\textbf{Real Context:} In a rural pharmacy in Cameroon, manage medicine stock.

\begin{itemize}
    \item Create \texttt{medicine_stock = ["Paracetamol", "Antibiotic", "Vitamins"]}.
    \item Add "Syrup" using \texttt{append}, remove "Antibiotic" using \texttt{remove}.
\end{itemize}

\begin{lstlisting}
medicine_stock = ["Paracetamol", "Antibiotic", "Vitamins"]
# Your code here...
\end{lstlisting}

\subsection{Exercise 10: Tuple for Dosages}
\begin{itemize}
    \item Declare \texttt{paracetamol_dosage = (500, "mg", "adult")}.
    \item Unpack and display; try a modification to see the error.
\end{itemize}

\subsection{Exercise 11: Dictionary for Details}
\begin{itemize}
    \item Create \texttt{medicine_details = {"name": "Paracetamol", "price": 500, "quantity": 100}}.
    \item Add "expiration": "2026-12" and iterate to display.
\end{itemize}

\begin{lstlisting}
medicine_details = {"name": "Paracetamol", "price": 500, "quantity": 100}
# Your code here...
\end{lstlisting}

\section{Integrative Project: Harvest Management System (30 min)}

\subsection{Instructions}
\textbf{Real Context:} A farmer tracks cocoa harvests over 5 days, calculates the average, identifies low-yield days, and stores the data in a dictionary.

\begin{itemize}
    \item Use a list for daily harvests (random 50-200 kg).
    \item Function for average.
    \item Loop with condition for days < 100 kg.
    \item Dictionary to associate day with harvest.
    \item Display if eligible for aid (average < 120 kg).
\end{itemize}

\begin{lstlisting}
import random

# Harvests list
harvests = [random.randint(50, 200) for _ in range(5)]

# Average function
def average_harvests(harvests):
    # Your code here...

# Analysis
results = {}
for i, harvest in enumerate(harvests, 1):
    results[f"Day {i}"] = harvest
    if harvest < 100:
        print(f"Day {i}: Low harvest ({harvest} kg)")

avg = average_harvests(harvests)
print(f"Average: {avg:.2f} kg")
if avg < 120:
    print("Eligible for agricultural aid")
print(results)
\end{lstlisting}

\subsection{Deliverable}
Save your complete Python codes with comments.

\textbf{Submission Procedure (Git):} 
\begin{enumerate}
    \item \textbf{Clone the repository:} \\ 
    \texttt{git clone https://github.com/club-genie-informatique-enspy/TP-PYTHON-TRAINING.git}
    \item \textbf{Create a branch:} Create a new branch for your work.
    \begin{itemize}
        \item \textbf{Naming Convention:} \texttt{tp\_python/tp1-<your-last-name>-<your-first-name>} (e.g., \texttt{tp\_python/tp1-delmat-azangue})
        \item \textbf{Command:} \texttt{git checkout -b tp\_python/tp1-<name>-<firstname>}
    \end{itemize}
    \item \textbf{Commit:} Stage, commit and add a clear message (e.g., \texttt{git commit -m "tp\_python(tp1): add harvest management system"}).
    \item \textbf{Push:} Send your branch to the remote repository.
    \item \textbf{Finalization:} Create a \textbf{Pull Request (PR)} to the \texttt{main} branch for review.
\end{enumerate}

\newpage

\section*{⏳ Conclusion (15 min)}
\begin{itemize}
    \item \textbf{Review:} Share results, discuss real applications (e.g., agriculture in Cameroon).
    \item \textbf{Feedback:} Note what was difficult (e.g., complex logic).
    \item \textbf{Preparation:} Move on to Week 4 for I/O and robustness.
\end{itemize}

This TP uses real scenarios to ground the concepts in practice.

\end{document}
