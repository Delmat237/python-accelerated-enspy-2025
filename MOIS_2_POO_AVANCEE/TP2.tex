\documentclass[12pt]{article}
\usepackage[utf8]{inputenc}
\usepackage[T1]{fontenc}
\usepackage[a4paper, margin=1in]{geometry}
\usepackage{fancyhdr}
\usepackage{amsmath}
\usepackage{enumitem}
\usepackage{listings}
\usepackage{xcolor}

% Configuration des listings pour le code Python
\lstset{
    language=Python,
    basicstyle=\ttfamily\small,
    keywordstyle=\color{blue},
    stringstyle=\color{red},
    commentstyle=\color{green!50!black},
    numbers=left,
    numberstyle=\tiny,
    stepnumber=1,
    numbersep=5pt,
    showspaces=false,
    showstringspaces=false,
    frame=single,
    breaklines=true,
    breakatwhitespace=true,
    tabsize=4
}

% En-tête et pied de page
\pagestyle{fancy}
\fancyhf{}
\fancyhead[L]{TP 2 - Mois 2 : POO et Architecture Logicielle}
\fancyhead[R]{21 décembre 2025}
\fancyfoot[C]{\thepage}

% Titre
\title{\textbf{TP Intégrateur - Programmation Orientée Objet (POO) et Tests Unitaires}}
\author{}
\date{}

\begin{document}

\maketitle

\section*{ Objectifs du TP (3h)}

Ce TP évalue votre capacité à concevoir une architecture logicielle robuste en utilisant les piliers de la POO : Encapsulation, Héritage, Polymorphisme. Vous devrez également valider votre code par des tests unitaires automatisés.

\section{ Encapsulation : Gestion d'Hôpital (45 min)}
\textbf{Contexte :} Concevez une classe \texttt{Patient} pour un hôpital à Yaoundé.

\begin{itemize}
    \item Attributs privés : \texttt{\_\_id}, \texttt{\_\_nom}, \texttt{\_\_dossier\_medical} (liste).
    \item Utilisez \texttt{@property} pour accéder au nom.
    \item Implémentez un setter pour le nom qui vérifie qu'il n'est pas vide.
    \item Méthode \texttt{ajouter\_diagnostic(info)} pour remplir le dossier.
\end{itemize}

\begin{lstlisting}
class Patient:
    def __init__(self, id_patient, nom):
        # Votre code ici...
\end{lstlisting}

\section{ Héritage et Polymorphisme : Flotte de Transport (1h)}
\textbf{Contexte :} Une agence de voyage camerounaise gère différents types de véhicules.

\begin{itemize}
    \item Classe Mère : \texttt{Vehicule} avec méthode \texttt{calculer\_tarif(distance)}.
    \item Sous-classes : \texttt{Bus} et \texttt{VoitureVIP}.
    \item \texttt{Bus} : Tarif fixe de 500 FCFA/km.
    \texttt{VoitureVIP} : Tarif de 1200 FCFA/km + service de bord (2000 FCFA).
    \item Utilisez \texttt{super()} dans les constructeurs.
\end{itemize}

\section{ Méthodes de Classe et Statiques : Gestion Financière (30 min)}
\textbf{Contexte :} Créez une classe \texttt{CompteBancaire}.

\begin{itemize}
    \item Attribut de classe : \texttt{taux\_interet = 0.03}.
    \item Méthode de classe : \texttt{modifier\_taux(nouveau\_taux)}.
    \item Méthode statique : \texttt{est\_numero\_valide(numero)} (vérifie si 10 chiffres).
\end{itemize}

\section{ Tests Unitaires : Validation des Calculs (45 min)}
\textbf{Contexte :} Testez votre système de transport avec \texttt{pytest}.

\begin{itemize}
    \item Créez une fixture pour un \texttt{Bus}.
    \item Testez le calcul du tarif pour une distance normale.
    \item Testez que le tarif ne peut pas être négatif (levez une \texttt{ValueError}).
\end{itemize}

\section{ Projet Final : Système de Gestion de Bibliothèque (1h)}
\textbf{Consigne :}
Créez un système avec :
1. Une classe \texttt{Livre} (Encapsulation).
2. Une classe \texttt{LivreNumerique} qui hérite de \texttt{Livre}.
3. Une classe \texttt{Bibliotheque} qui gère une liste de livres.
4. Un fichier de test \texttt{test\_bibliotheque.py} validant l'ajout et l'emprunt.

\textbf{Submission Procedure (Git) :} 
Suivre la même procédure que le TP1 en utilisant la branche \texttt{tp\_python/tp2-<nom>-<prenom>}.

\end{document}
