\documentclass[12pt]{article}
\usepackage[utf8]{inputenc}
\usepackage[T1]{fontenc}
\usepackage[a4paper, margin=1in]{geometry}
\usepackage{fancyhdr}
\usepackage{amsmath}
\usepackage{enumitem}
\usepackage{listings}
\usepackage{xcolor}

% Configuration for Python code listings
\lstset{
    language=Python,
    basicstyle=\ttfamily\small,
    keywordstyle=\color{blue},
    stringstyle=\color{red},
    commentstyle=\color{green!50!black},
    numbers=left,
    numberstyle=\tiny,
    stepnumber=1,
    numbersep=5pt,
    showspaces=false,
    showstringspaces=false,
    frame=single,
    breaklines=true,
    breakatwhitespace=true,
    tabsize=4
}

% Header and Footer
\pagestyle{fancy}
\fancyhf{}
\fancyhead[L]{TP 2 - Month 2: OOP and Software Architecture}
\fancyhead[R]{December 21, 2025}
\fancyfoot[C]{\thepage}

% Title
\title{\textbf{Integrative TP - Object-Oriented Programming (OOP) and Unit Testing}}
\author{}
\date{}

\begin{document}

\maketitle

\section*{TP Objectives (3h)}

This TP evaluates your ability to design a robust software architecture using the pillars of OOP: Encapsulation, Inheritance, Polymorphism. You will also have to validate your code through automated unit tests.

\section{Encapsulation: Hospital Management (45 min)}
\textbf{Context:} Design a \texttt{Patient} class for a hospital in Yaoundé.

\begin{itemize}
    \item Private attributes: \texttt{\_\_id}, \texttt{\_\_name}, \texttt{\_\_medical\_record} (list).
    \item Use \texttt{@property} to access the name.
    \item Implement a setter for the name that checks that it is not empty.
    \item Method \texttt{add\_diagnosis(info)} to fill the record.
\end{itemize}

\begin{lstlisting}
class Patient:
    def __init__(self, patient_id, name):
        # Your code here...
\end{lstlisting}

\section{Inheritance and Polymorphism: Transport Fleet (1h)}
\textbf{Context:} A Cameroonian travel agency manages different types of vehicles.

\begin{itemize}
    \item Parent Class: \texttt{Vehicle} with \texttt{calculate\_fare(distance)} method.
    \item Sub-classes: \texttt{Bus} and \texttt{VIPCar}.
    \item \texttt{Bus}: Fixed rate of 500 FCFA/km.
    \texttt{VIPCar}: Rate of 1200 FCFA/km + on-board service (2000 FCFA).
    \item Use \texttt{super()} in constructors.
\end{itemize}

\section{Class and Static Methods: Financial Management (30 min)}
\textbf{Context:} Create a \texttt{BankAccount} class.

\begin{itemize}
    \item Class attribute: \texttt{interest\_rate = 0.03}.
    \item Class method: \texttt{change\_rate(new\_rate)}.
    \item Static method: \texttt{is\_valid\_number(number)} (checks if 10 digits).
\end{itemize}

\section{Unit Testing: Calculation Validation (45 min)}
\textbf{Context:} Test your transport system with \texttt{pytest}.

\begin{itemize}
    \item Create a fixture for a \texttt{Bus}.
    \item Test the fare calculation for a normal distance.
    \item Test that the fare cannot be negative (raise a \texttt{ValueError}).
\end{itemize}

\section{Final Project: Library Management System (1h)}
\textbf{Instructions:}
Create a system with:
1. A \texttt{Book} class (Encapsulation).
2. A \texttt{DigitalBook} class that inherits from \texttt{Book}.
3. A \texttt{Library} class that manages a list of books.
4. A test file \texttt{test\_library.py} validating adding and borrowing.

\textbf{Submission Procedure (Git):} 
Follow the same procedure as TP1 using the branch \texttt{tp\_python/tp2-<name>-<firstname>}.

\end{document}
